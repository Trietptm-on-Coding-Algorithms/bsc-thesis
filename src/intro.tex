\chapter{Introduction}

Binary analysis is one of the most important problems in computer security.
Despite the rise of web applications and interpreted languages like Python, the are several reasons that justify this importance, as described in \cite{shoshitaishvili2016state}.
Nowadays, operating systems are still based on compiled languages and also interpreted languages are often compiled down during execution by a Just In Time compiler. In addition, traditional low-level languages like C are reborn thanks to the rise of Internet Of Things and the associated resource-constrained devices.

A wide family of manual and automatic techniques has been developed over the years to analyze this kind of programs.
These techniques are divided into static and dynamic analyses depending on the need to execute the program or not.

Despite the continuous innovation on the front of the automatic analysis, the human part remains essential.
A reverse engineer is a person who tries to understand what a binary program does and how.
This process usually involves reading and understanding the disassembled code and its effects during execution with the aid of a debugger.

One of the most used automatic techniques nowadays is {\em Symbolic Execution}.
The idea came about in the '70s \cite{King} but only recently its applications became relevant in computer security.
Symbolic execution is used for different tasks, varying from deobfuscation to vulnerability detection and automatic exploit generation.

Symbolic execution has been traditionally considered a static analysis technique since in its original embodiment the application code is not executed concretely by the CPU. However, a more recent twist of symbolic execution, known as Dynamic Symbolic Execution (DSE) \cite{DART}, uses a concrete execution to drive the symbolic exploration and thus can be classified as a dynamic analysis technique.

In the last years, symbolic execution becomes also a first-class technique used by anyone who deals with manual reverse engineering.

% explain the connection with reversing here

Execute symbolically an entire complex software (like a web server for example) is a huge task for a machine, even for a supercomputer so the analyst often uses it in a surgical manner on small pieces of code.
For instance, a reverse engineer often needs to reverse custom encryption functions or obfuscated code during the dynamic analysis.

In this thesis, we introduce the idea of combining symbolic execution with dynamic analysis for reverse engineering.

Differently from DSE, we devise an approach where the reverse engineer can use a debugger to drive and inspect a concrete execution engine of the application code and then, when needed, transfer the execution into a symbolic executor in order to automatically identify the input values required to reach a target point in the code. After that, the user can also transfer back the correct input values found with symbolic execution in order to continue the debugging.

The synchronization between a debugger and a symbolic executor can enhance manual dynamic analysis and allow a reverser to easily solve small portions of code without leaving the debugger.

We implemented a synchronization mechanism on top of the binary analysis framework {\em angr}, allowing for transferring the state of the debugged process to the angr environment and back.

The backend library is debugger agnostic and can be extended to work with various frontends.
We implemented a frontend for the {\em IDA Pro} debugger and one for the {\em GNU Debugger}, which are both widely popular among reverse engineers.

\subsubsection{Structure of the Thesis}

Chapter 1 describes what is symbolic execution and its limitations. Chapter 2 presents our technique and discuss advantages and limitations. Chapter 3 is about the implementation of the technique in a library based on angr, {\em AngrDBG}, and its frontends, {\em AngrGDB} and {\em IDAngr}. Chapter4 is a usage example of IDAngr to show how the tool works.
